\documentclass[a4paper,12pt]{article}

\usepackage[T1,T2A]{fontenc}
\usepackage[utf8]{inputenc}
\usepackage[utf8]{inputenc}
\usepackage[russian]{babel}
\usepackage{amsmath}
\begin{document}
\par Для решения это задачи можно применить двоичный поиск по ответу. Двоичный поиск будет обрабатывать длину стороны $k$-угольника, так как площадь правильного $k$-угольника прямо зависит от его стороны (формула площади правильного $k$-угольника со стороной $a$ --- $\frac{k}{4}a^2\ctg \frac{\pi}{k}$). Правой границей установим $\max_{a_i} {a_1, \ldots, a_n}$, так как большую сторону, чем максимальную мы не возьмем, так как одна сторона содержит только одну палочку, а левой $0$, т.е. несущетвование $k$-угольника. Двоичный поиск по ответу здесь работает, так как функция существования правильного $k$-угольника со стороной $x$ в данном случае не возрастающая, т.к. если мы можем собрать многоугольник с длиной стороны $x$, то очевидно мы можем собрать многоугольник с длиной стороны $y$ такой, что $y < x$. Значит что если при каком-то длине стороне правильный $k$-угольник можно собрать из данных палочек, значит, что из данных палочек можно собрать правильные $k$-угольник всех меньших сторон, следовательно функция на некотором префиксе функция будет возвращать $1$ (т.е. можно собрать такой $k$-угольник), а дальше будет только $0$ (нельзя собрать из данных палочек), значит двоичный поиск для этой функции будет работать.\parАсимптотика решения: $\mathcal{O}(n \log \max_{a_i} {a_1, \ldots, a_n})$.
\end{document}