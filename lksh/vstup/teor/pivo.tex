\documentclass[a4paper,12pt]{article}

\usepackage[T1,T2A]{fontenc}
\usepackage[utf8]{inputenc}
\usepackage[utf8]{inputenc}
\usepackage[russian]{babel}
\usepackage{amsmath}
\begin{document}
\par Для решения этой задачи можно использовать декартово дерево. Вставка и удаление будут реализованы так же, как и в декартовом дереве с явным ключом. \parА вместо последней операции реализуем такую, которая просто находить минимальный элемент, которого нет в множестве. Для того чтобы эту операцию свести к исходной, нужно применить операцию $split$ (причем реализация этой функции должна быть, такая, в которой вершины со ключом $x$ находятся в правом поддереве) к исходному дереву, деля по ключу, который является аргументом. Тогда если применить нашу функцию нахождения минимального элемента, которого нет в множестве, к правому поддереву, то получим нужный результат --- минимальный элемент, меньший либо равный $x$, которого нет в множестве. Реализация этой функции будет такова:
\begin{enumerate}
\item Реализация будет рекурсивной, в аргументах будет вершина дерева, которую сейчас обрабатываем и указатель на минимальный элемент, которого нет в множестве.
\item Изначально будем считать, что ответ --- $x$, а первой обрабатываемой вершиной будет корень поддерева, где все ключи будут большие $x$.
\item Если мы пытаемся перейти в несуществующую вершину, то значит, что найден ответ, вернем указатель на ответ.
\item Иначе: 
\begin{enumerate}
\item Если в левом поддереве не содержится все ключи, которые принадлежат отрезку $[x;$ключ обрабатываемой вершины$]$ (Его размер не равен ключ - указатель на минимальный ответ), то перейдем в левое поддерево, не передвигая указатель.
\item Иначе, перейдем в правое поддерево, переместив указатель минимального возможного ответа на ключ обрабатываемой вершины $+ 1$.
\end{enumerate}
\end{enumerate}
\end{document}